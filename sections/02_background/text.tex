Foundations for terms and concepts of generalization, deep learning as a form of artificial learning and generalization in deep learning will be formulated in this chapter.

\section{Learning, Artificial Learning and Generalization}

    % https://books.google.de/books?hl=de&lr=&id=qOF4AgAAQBAJ&oi=fnd&pg=PP1&ots=vNTiX5Lu_U&sig=UEC7rljJ7dVsAbh_XQeMmCrNqcU#v=onepage&q&f=false
    The principal mechanisms of learning were studied in the last century by investigating conditioned learning, where an outcome is associated to a stimulus by a learning organism e.g. a dog that awaits food after hearing the sound of a bell \citep{pavlov1928conditioned, pavlov2010conditioned, banich2011generalization}. Later, fear responses were especially studied and two sub-mechanisms of conditioned learning --- generalization and specialization --- were discovered \citep{banich2011generalization}.
    In fear learning, initially an instance-based generalization occurs that maps a novel fear to an environment \citep{banich2011generalization}. Later, this generalization is specialized and mapped to specific environmental stimuli leading to discrimination \citep{banich2011generalization}.
    It was discovered, that generalization can occur intra-modal and cross-modal for the example of the food awaiting dog either receiving visual or auditory stimuli \citep{pavlov1928conditioned} and that gradients of generalization exist \citep{guttman1956discriminability}.
    The concept of gerneralization and spezialization can be tracked down to individual parts of the brain, where the initial generalized learning is associated to the amygdala whereas the specialization occurs in the prefrontal cortex and the hippocampus \citep{banich2011generalization}.

    \subsection{Deep Learning}
        how does deep learning work?
        what are the challenges for generalization in artificial and deep learning?
        How is that equal to learning and generalization in learning, where does it differ?

    \subsection{Generalization in Deep Learning: Current discussion and challenges}
        What are the current approaches in deep learning? How is the idea of AGI related to our gerneralization approach? What can be problems of AGI and why do we analyse generalization in a limited scope such as medical image analysis?



    And now that we know how artificial learning works and what generalization is, we can transfer this knowledge to medical deep learning.

\section{Generalization in the Medical Image Analysis Pipeline}
    What special properties apply for medical imaging?
    What makes Generalization hard?
    What can be done to achieve generalization in each of the steps?

    \subsection{Data Acquisition}
    \subsection{Data Curation and Preprocessing}
        \subsubsection{Modalities}
        \subsubsection{Scanners}
            Histograms
    \subsection{Model Building}
        \subsubsection{Convolutional Models}
            Kernels

    \subsection{Model Application and Evaluation}

    Now that we know what is special about medical deep learning and what has been done in the field, we know what is possible and missing and this is our starting point to present our methods in the context of these methods.