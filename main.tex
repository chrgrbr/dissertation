\documentclass[
  11pt,
  a4paper,
  twoside,
  headsepline,
  titlepage,
  parskip=off
  DIV=11,
  BCOR=12mm,
  captions=tableheading,
  chapterprefix=on,
  numbers=noenddot
]{scrbook}

% Packages
\usepackage[utf8]{inputenc}
\usepackage[T1]{fontenc}
\usepackage{lmodern}
\usepackage{csquotes}
\addtokomafont{disposition}{\rmfamily}
\usepackage[english]{babel}
\usepackage{amssymb, amsmath}
\usepackage[pdftex]{graphicx}
\usepackage{placeins}
\usepackage{xspace}
\usepackage{booktabs,adjustbox}
\usepackage{multirow}
\usepackage{mathtools}
\usepackage[font=small,labelfont={bf}]{caption}
\usepackage{nomencl}
  \let\abbrev\nomenclature
  \renewcommand{\nomname}{Abkürzungsverzeichnis}
  \setlength{\nomlabelwidth}{.15\hsize}
  \renewcommand{\nomlabel}[1]{#1 }%\dotfill}
  \setlength{\nomitemsep}{-\parsep}
  \makenomenclature
\usepackage[normalem]{ulem}
  \newcommand{\markup}[1]{\uline{#1}}
\usepackage{chngcntr}
\counterwithout{footnote}{chapter}
\usepackage{tabularx}
\usepackage{lipsum}
\usepackage[letterspace=-80]{microtype}
\usepackage{bm}
\usepackage{tikz}
\usetikzlibrary{positioning,arrows.meta}
\tikzdeclarecoordinatesystem{timeline}{% #1 is the date in years
    \pgfmathsetmacro\myx{(#1-2018)*3}
    \pgfpointxy{\myx}{0}
}
\usetikzlibrary{calc}
\usepackage{environ}
\makeatletter
\newsavebox{\measure@tikzpicture}
\NewEnviron{scaletikzpicturetowidth}[1]{%
  \def\tikz@width{#1}%
  \def\tikzscale{1}\begin{lrbox}{\measure@tikzpicture}%
  \BODY
  \end{lrbox}%
  \pgfmathparse{#1/\wd\measure@tikzpicture}%
  \edef\tikzscale{\pgfmathresult}%
  \BODY
}
\makeatother
\def\centerarc[#1](#2)(#3:#4:#5)% Syntax: [draw options] (center) (initial angle:final angle:radius)
    { \draw[#1] ($(#2)+({#5*cos(#3)},{#5*sin(#3)})$) arc (#3:#4:#5); }
\usepackage[ruled]{algorithm2e}
\usepackage{subcaption}
\usepackage{setspace}
\usepackage[figuresright]{rotating}
\usepackage{placeins}
\usepackage{siunitx}
\usepackage{bigdelim}
\usepackage{bbm}

\usepackage{enumitem}
\setlist[itemize]{leftmargin=*}
\setlist[enumerate]{leftmargin=*}

\linespread{1.05}
\typearea[current]{last}

\usepackage[pdftex,
    bookmarks,
    bookmarksopen,
    bookmarksnumbered,
    colorlinks,
    linkcolor=black,
    citecolor=black,
    allcolors=black,
    pdfstartview=FitH,
    plainpages=false,
    %pdfpagelabels,
    hypertexnames=false
    ]{hyperref}
\usepackage[final]{pdfpages}

\hypersetup{pdfprintscaling=None}

\usepackage[
backend=biber,
citestyle=authoryear-comp,
bibstyle=authortitle,
isbn=false,
url=false,
doi=false,
maxbibnames=99,
maxcitenames=1,
uniquelist=false,
sorting=anyt,
natbib=true,
giveninits=true,
uniquename=false,
maxalphanames=1,labelalpha,
citetracker=true,
defernumbers=true,
dashed=false
]{biblatex}

\defbibheading{subbibliography}[List of Publications]{%
  \addsec*{#1}%
  \markboth{List of Publications}{#1}% DELETED
  }% NEW

\addbibresource{bib/references.bib}
% \addbibresource{bib/journals_first_author.bib}
% \addbibresource{bib/proceedings_first_author.bib}
% \addbibresource{bib/proceedings_co_author.bib}

\setlength{\bibitemsep}{0.25em}     % Abstand zwischen den Literaturangaben
\setlength{\bibhang}{3em}        % Einzug nach jeweils erster Zeile

\preto\fullcite{\AtNextCite{\defcounter{maxnames}{99}}}

\AtEveryBibitem{\clearname{editor}}
\AtEveryBibitem{\clearlist{organization}}
\AtEveryBibitem{\clearlist{location}}
\AtEveryBibitem{\clearfield{month}}
\AtEveryCitekey{\clearname{editor}}
\AtEveryCitekey{\clearlist{organization}}
\AtEveryCitekey{\clearlist{location}}
\AtEveryCitekey{\clearfield{series}}
\AtEveryCitekey{\clearfield{month}}
\newif\ifPubList
\AtEveryBibitem{\renewcommand*{\mkbibnamegiven}[1]{\ifPubList\ifitemannotation{bold}{\textbf{#1}}{#1}\else{#1}\fi\ifitemannotation{equal}{\textsuperscript{\textbf{*}}}}}
\AtEveryBibitem{\ifPubList\renewcommand*{\mkbibnamefamily}[1]{\ifitemannotation{bold}{\textbf{#1}}{#1}}\fi}
\AtEveryCite{\renewcommand*{\mkbibnamegiven}[1]{\ifitemannotation{equal}{{#1}\textsuperscript{\textbf{*}}}{#1}}}
\DeclareNameAlias{author}{family-given}
\renewbibmacro{in:}{\ifentrytype{article}{}{\printtext{\bibstring{in}\intitlepunct}}}

\renewbibmacro*{volume+number+eid}{%
  \printfield{volume}%
  \setunit*{\addnbthinspace}%
  \printfield{number}%
  \setunit{\addcomma\space}%
  \printfield{eid}}
\DeclareFieldFormat[article]{number}{\mkbibparens{#1}}

\DeclareSortingTemplate{ymdnt}{
  \sort{
    \field{presort}
  }
  \sort[final]{
    \field{sortkey}
  }
  \sort[direction=descending]{
    \field[strside=left,strwidth=4]{sortyear}
    \field[strside=left,strwidth=4]{year}
    \literal{9999}
  }
  \sort[direction=descending]{
    \field{month}
    \literal{9999}
  }
  \sort{
    \field{sortname}
    \field{author}
    \field{editor}
    \field{translator}
    \field{sorttitle}
    \field{title}
  }
  \sort{
    \field{sorttitle}
    \field{title}
  }
}

\makeatletter

\newrobustcmd*{\parentexttrack}[1]{%
  \begingroup
  \blx@blxinit
  \blx@setsfcodes
  \blx@bibopenparen#1\blx@bibcloseparen
  \endgroup}

\AtEveryCite{%
  \let\parentext=\parentexttrack%
  \let\bibopenparen=\bibopenbracket%
  \let\bibcloseparen=\bibclosebracket}

\makeatother

\newcounter{mymaxcitenames}
\AtBeginDocument{%
  \setcounter{mymaxcitenames}{\value{maxnames}}%
}

\renewbibmacro*{begentry}{%
  \printtext[brackets]{%
    \begingroup
    \defcounter{maxnames}{\value{mymaxcitenames}}%
    \printnames{labelname}%
    \setunit{\nameyeardelim}%
    \usebibmacro{cite:labeldate+extradate}%
    \endgroup
    }
\enspace
}

\renewbibmacro*{issue+date}{%
  \setunit{\addcomma\space}% NEW
%  \printtext[parens]{% DELETED
    \iffieldundef{issue}
      {\usebibmacro{date}}
      {\printfield{issue}%
       \setunit*{\addspace}%
%       \usebibmacro{date}}}% DELETED
       \usebibmacro{date}}% NEW
  \newunit}

\DeclareNameAlias{sortname}{first-last}

\defbibenvironment{mypubs2}
  {\enumerate[widest=50]
     {}
     {\setlength{\labelwidth}{\labelnumberwidth}%
      \setlength{\leftmargin}{\labelwidth}%
      \setlength{\labelsep}{\biblabelsep}%
      \addtolength{\leftmargin}{\labelsep}%
      \setlength{\itemsep}{\bibitemsep}%
      \setlength{\parsep}{\bibparsep}}%
      \renewcommand*{\makelabel}[1]{\hss##1}}
  {\endenumerate}
  {\item}

\defbibenvironment{mypubs1}
  {\itemize
     {}
     {\setlength{\labelwidth}{\labelnumberwidth}%
      \setlength{\leftmargin}{\labelwidth}%
      \setlength{\labelsep}{\biblabelsep}%
      \addtolength{\leftmargin}{\labelsep}%
      \setlength{\itemsep}{\bibitemsep}%
      \setlength{\parsep}{\bibparsep}}%
      \renewcommand*{\makelabel}[1]{\hss##1}}
  {\enditemize}
  {\item[--]}


\defbibcheck{mynocitejourn}{%
  \ifboolexpr{test {\ifkeyword{ownjourn}}}
    {}
    {\skipentry}
}

\defbibcheck{mynociteconf}{%
  \ifboolexpr{test {\ifkeyword{ownconf}}}
    {}
    {\skipentry}
}

\usepackage{xcolor}



% Layoutmodifikationen
\renewcommand{\arraystretch}{1.1}
\setkomafont{captionlabel}{\bfseries\textsf\sffamily}
\setkomafont{caption}{\textsf\sffamily}
\setcounter{secnumdepth}{3}
\setcounter{tocdepth}{2}
\renewcaptionname{english}{\figurename}{Fig.}

%% FARBEN:
\newcommand\crule[3][black]{\textcolor{#1}{\rule{#2}{#3}}}
\definecolor{uzl_ocean}{cmyk}{1,.0,.20,.78} % RGB 0 75 90
\definecolor{uzl_gray1}{RGB}{80,105,121} % 34 13 0 53
\definecolor{uzl_gray2}{RGB}{101,101,101} % 0 0 0 60
\definecolor{uzl_gray3}{RGB}{117,115,105} % 0 2 10 54
\definecolor{gst_orange}{RGB}{245,154,0} % 0 37 100 4
% blue rgb: 60 169 213, cmyk: 72 21 0 16
% red  rgb: 181 22 33,  cmyk: 0 88 82 29
\definecolor{Ocean}{cmyk}{1,0,0.2,0.78}
\definecolor{Grey}{cmyk}{0,0,0,0.6}
\definecolor{ULGray}{cmyk}{0,0,0.03,0.1}
\definecolor{mygreen}{RGB}{0,127,0}
\definecolor{myred}{RGB}{255,0,0}
\definecolor{myblue}{RGB}{0,0,255}
\definecolor{rightkidney}{RGB}{255,232,0}
\definecolor{liver}{RGB}{110,15,202}
\definecolor{stomach}{RGB}{251,14,255}
\definecolor{leftkidney}{RGB}{63,182,73}
\definecolor{esophagus}{RGB}{0,114,255}
\definecolor{spleen}{RGB}{210,2,0}
\definecolor{pancreas}{RGB}{46,175,181}
\definecolor{aorta}{RGB}{170,107,0}
\definecolor{venacava}{RGB}{254,161,68}
\definecolor{gall}{RGB}{0,252,124}
\definecolor{lag}{RGB}{107,0,208}
\definecolor{rag}{RGB}{62,27,203}
\definecolor{portal_vein}{RGB}{0,132,32}
\definecolor{legend_pcdd}{RGB}{0,176,240}
\definecolor{legend_pcdmed}{RGB}{112,48,160}
\definecolor{legend_single}{RGB}{169,209,142}
\definecolor{legend_sequence}{RGB}{56,87,35}
\definecolor{legend_ind}{RGB}{255,192,0}
\definecolor{legend_da}{RGB}{191,144,0}
\newcommand{\tikzcircle}[2][red,fill=red]{\tikz[baseline=-0.5ex]\draw[#1,radius=#2] (0,0) circle ;}
\DeclareRobustCommand{\legendsquare}[1]{%
  \textcolor{#1}{\rule{1.5ex}{1.5ex}}%
}
%schusterjungen und hurenkinder vermeiden
\clubpenalty=10000
\widowpenalty=10000
\displaywidowpenalty=10000

\DeclareMathAlphabet{\mb}{OT1}{cmr}{bx}{it}

%% my definitions
\newcommand{\mB}[1]{\mathbf{#1}}
\newcommand{\mspan}[1]{\mathrm{span}(#1)}
\newcommand{\todo}[1]{\textbf{\textcolor{red}{#1}}} %creates a red markup
\newcommand{\norm}[1]{\left\lVert#1\right\rVert}
\newcommand{\zB}{\mbox{z.\,B.}\xspace}
\newcommand{\ua}{\mbox{u.\,a.}\xspace}
\newcommand{\uU}{\mbox{u.\,U.}\xspace}
\newcommand{\printpublication}[1]{\AtNextCite{\defcounter{maxnames}{99}}\fullcite[][]{#1}}

\recalctypearea

\newlength{\displacement}
\setlength{\displacement}{3mm}
\addtolength{\oddsidemargin}{\displacement}
\addtolength{\evensidemargin}{-\displacement}
\addtolength{\marginparwidth}{-\displacement}

\def\makenamesetup{%
\def\bibnamedelima{~}%
\def\bibnamedelimb{ }%
\def\bibnamedelimc{ }%
\def\bibnamedelimd{ }%
\def\bibnamedelimi{ }%
\def\bibinitperiod{.}%
\def\bibinitdelim{~}%
\def\bibinithyphendelim{.-}}
\newcommand*{\makename}[2]{\begingroup\makenamesetup\xdef#1{#2}\endgroup}

\newcommand*{\boldname}[3]{%
  \def\lastname{#1}%
  \def\firstname{#2}%
  \def\firstinit{#3}}
\boldname{}{}{}

\boldname{Author}{Some}{S.}

\usepackage[normalem]{ulem}

\makeatletter
\newcommand{\namehighighter}[1]{%
  \ifboolexpr{(test {\ifdefequal{\firstname}{\namepartgiven}}
               or test {\ifdefequal{\firstinit}{\namepartgiven}})
              and test {\ifdefequal{\lastname}{\namepartfamily}}}
    {\uline{#1}}
    {#1}}

\renewbibmacro*{name:family-given}[4]{%
  \usebibmacro{name:delim}{#1#2#3}%
  \usebibmacro{name:hook}{#1#2#3}%
  \namehighighter{%
  \mkbibnamefamily{#1}\isdot\addcomma
  \ifdefvoid{#4}{}{\bibnamedelimd\mkbibnamesuffix{#4}\isdot}
  \ifdefvoid{#2}{}{\mkbibnamegiven{#2}\isdot}%
  }}
\makeatother

\newcolumntype{C}[1]{>{\centering\arraybackslash}p{#1}}
\renewcommand{\thealgocf}{}

\makeatletter
\renewcommand*{\@algocf@pre@ruled}{\hrule height 1.5pt depth0pt \kern\belowrulesep}
\renewcommand*{\algocf@caption@ruled}{\box\algocf@capbox\kern\aboverulesep\hrule height\lightrulewidth\kern\belowrulesep}
\renewcommand*{\@algocf@post@ruled}{\kern\aboverulesep\hrule height 1.5pt}
\makeatother

\makeatletter
% Remove right hand margin in algorithm
\patchcmd{\@algocf@start}% <cmd>
  {-1.5em}% <search>
  {0pt}% <replace>
  {}{}% <success><failure>
\makeatother

\DeclareMathOperator*{\avg}{avg}
\DeclareMathOperator*{\argmax}{arg\,max}
\DeclareMathOperator*{\argmin}{arg\,min}

\newcommand{\shared}[1]{\uline{Hermes*, N.}}

\pdfsuppresswarningpagegroup=1

\begin{document}
%% ------------------------------------------------------------------------
%% TITELSEITEN
%% ------------------------------------------------------------------------
\thispagestyle{headings}
\pagenumbering{roman}
\begin{titlepage}
\vspace*{\stretch{1}}

\addtolength{\topmargin}{-1.2cm}
%\addtolength{\hoffset}{-1.57cm}
\addtolength{\textwidth}{2.35cm}
%\setlength{\footskip}{0cm}

% ------------------------------------------------------------
\vspace*{-2.7cm}
% \includegraphics[width=0.65\textwidth]{./images/logo_imi_en.pdf}
\vspace*{0.4cm}
\begin{center}

% ------------------------------------------------------------

\enlargethispage{5cm}
\textbf{From the Institute of Medical Informatics}\\
\textbf{of the University of Lübeck}\\
\textbf{Director: Prof. Dr. rer. nat. habil. Heinz Handels}\\[2.8cm]
%
%% ---------------------------------------------------------------------------
%% Schriftgröße des Titels sollte an die Länge des Titels angepasst werden...
%% ---------------------------------------------------------------------------
\begin{Large}
%\textbf{Effiziente Schätzung der Atembewegung mithilfe von statistischen Bewegungs"= und Korrespondenzmodellen}\\ %[0.35cm] %unter Verwendung
\textbf{Geometric Deep Learning for Hand Skeleton Tracking and Gesture Recognition on 3D Point Clouds}\\ %[0.35cm] %unter Verwendung
% Geometric Deep Learning on 3D Point Clouds for Medical Image Analysis
%title: Prior-guided 3D Deep Learning on Point Clouds in Medicine
%Prior-guided geometric deep learning for medical point cloud analysis under domain shifts.
%Deep learning-based point cloud analysis in medicine: Bridging the gap through prior domain knowledge
%\textbf{Modellbasierte Schätzung der Atembewegung in nichtinvasiven Tumortherapieverfahren}\\ %[0.35cm] %unter Verwendung
%\textbf{Intrainterventionelle Schätzung der Atembewegung mittels modellbasierter Verfahren}\\ %[0.35cm] %unter Verwendung
%\textbf{Modellbasierte Schätzung der Atembewegung während nicht"=invasiver Strahlentherapie und Thermoablation}\\ %[0.35cm] %unter Verwendung
%\textbf{Intrainterventionelle Schätzung der Atembewegung mittels patientenspezifischer und populationsbasierter Modelle}\\ %[0.35cm] %unter Verwendung räumlich-zeitlicher Bilddaten
\end{Large}
\vspace*{2.5cm}
%
%
Dissertation\\
for\\
Fulfillment of Requirements for the Doctoral Degree\\
of the University of Lübeck\\[1.0cm]
%
from the Department of Computer Sciences and Technical Engineering\\[1.0cm]
%
Submitted by\\[0.1cm]
Christian Weihsbach\\[0.1cm]
from Minden\\[3.0cm]
%
Lübeck, 2024
\end{center}

% ------------------------------------------------------------

%\hspace*{11.0cm}
%\includegraphics[width=0.28\textwidth]{Slogan_Uni_Luebeck_CMYK}

% ------------------------------------------------------------

\vspace*{\stretch{5}}
\newpage
\thispagestyle{empty}

\addtolength{\topmargin}{1.2cm}
%\addtolength{\hoffset}{1.57cm}
\addtolength{\textwidth}{-2.35cm}

\end{titlepage}

\begin{titlepage}
\addtolength{\topmargin}{18cm}
\noindent First referee:\\
Second referee:\\\\
Date of oral examination:\\\\
Approved for printing. Lübeck,
\end{titlepage}

\chapter*{Abstract}
\enlargethispage{1\baselineskip}
\pagenumbering{roman}
\setcounter{page}{1}
This is an abstract.

\chapter*{Zusammenfassung}
Und dies ist die Zusammenfassung.
% Automated computational image analysis offers promising opportunities for healthcare, such as assisting clinicians in patient monitoring and medical scan assessment.
% Over the past decade, deep learning has significantly advanced the field, especially the analysis of dense intensity data on regular grids.
% By contrast, processing sparse 3D point clouds, captured by depth sensors in monitoring systems or extracted as descriptive keypoints from medical 3D scans, has received little attention yet.
% Given the potential benefits of the sparse geometric representation, including anonymity preservation, robustness to intensity variations, and computational efficiency, geometric deep learning on point clouds has great potential to play an increasingly important role in medical image analysis.
% However, various research questions, such as solving medicine-specific problems and investigating and tackling domain gaps, remain underexplored compared to dense image processing.
% %, impeding the integration of geometric deep learning into clinical practice.
% This thesis addresses these shortcomings in three ways.
%
% First, various point cloud-based deep learning methods for a heterogeneous set of medical tasks are developed, namely dynamic hand gesture recognition and in-bed body weight and pose estimation as monitoring tasks, and lung registration as a classical medical image analysis problem.
% The investigated tasks thus cover local detection and global regression/classification problems while considering single frames and temporal sequences as inputs.
% Promising experimental results across all these tasks demonstrate the versatile potential of point cloud-based approaches in medicine.
%
% Second, the thesis investigates the impact of domain gaps, revealing a significant sensitivity of point cloud networks to geometric domain shifts, and, in response, develops multiple novel domain adaptation strategies.
% An anatomy-guided constrained optimization scheme is presented for pose estimation, and the Mean Teacher paradigm is adapted and significantly extended for domain adaptive point cloud registration.
% Both approaches improve upon the current state of the art in comprehensive evaluations.
%
% Third, the thesis explores how solving the above problems can benefit from incorporating task-specific prior knowledge in various forms.
% On the one hand, the work derives an improved point cloud encoding scheme and a novel loss function from priors on the input and output distribution.
% On the other, it draws inspiration from human behavior and reasoning to develop novel learning paradigms and suitable two-stage and two-stream model architectures.
% Thorough experiments demonstrate significant advantages of these solutions over end-to-end approaches without explicit priors.
%
% Overall, the developed geometric deep learning models and domain adaptation strategies successfully address diverse problems in medical image processing and thus significantly contribute to advancing point cloud analysis in medicine.


\chapter*{Zusammenfassung}
Die automatisierte maschinelle Interpretation einer Handpose sowie die räumliche Modellierung der Hand halten ein vielversprechendes Potential für eine berührungslose Form der Interaktion zwischen Mensch und Maschine inne. Seit Jahrzenten werden neue Verfahren entwickelt und neue Sensoren entworfen, um diese intuitive Form der Interaktion zu ermöglichen. Durch den bahnbrechenden Erfolg von Deep Learning Verfahren hat sich der Fokus in den letzten Jahren vor allem auf bildverarbeitende Methoden gerichtet und insbesondere Bildmodalitäten mit regelmäßiger Gitterstruktur wurden intensiv erforscht. Die Hand als anatomisch äußerst komplexes Organ manifestiert sich durch die vielen Abhängigkeiten der Muskel- und Bandapparate im Allgemeinen als anspruchsvoll zu modellierende Struktur. Insbesondere in diesem Zusammenhang stellen geometrische Domänen wie Punktwolken oder Graphen eine

Im Fokus dieser Arbeit steht die Untersuchung neuer Möglichkeiten, die sich durch die Verarbeitung geometrischer Domänen wie Punktwolken oder Graphen anhand lernender Verfahren im Bereiche der Handanalyse und Interpretation menschlicher Gestik ergeben. In diesem Zusammenhang werden verschiedene Techniken und Methoden entwickelt und vorgestellt, die im allgemeinen dem Spektrum der geometrischen Deep-Learning Verfahren entspringen und eine Vielzahl von Anwendungen im Feld der Handanalyse abdecken.
Das Lernen auf Graphen wird sowohl für die Extraktion lokaler räumlicher Merkmale in Punktwolken untersucht, als auch um den globalen Kontext einer vorliegenden Struktur zu Analysieren und so die Lokalisation von Keypoints zu ermöglichen. In dem Kontext gilt es ebenfalls den Informationsfluss zwischen zwei semantisch unterschiedlichen Punktwolken, durch eine geeignete Strategie zur Verknüpfung der Knoten innerhalb des Graphen, sinnvoll zu gestalten.
Ein weiterer wichtiger Teil dieser Arbeit umfasst die Einbindung zeitlicher Informationen in Form einer Bewegungsanalyse in die Schätzung von Handposen. Hierzu wird eine neue Technik zur Optimierung von Verfahren zur Bestimmung des Szenenflusses für die spezifische Analyse der Bewegung des Handskeletts entwickelt. Dieser Bereich wird durch die Vorstellung eines neuen Deep-Learning Verfahrens vervollständigt, welches die Bewegungsinformationen einer Hand analysiert um die statische Schätzung einer Handpose zu präzisieren.
Im abschließenden methodischen Teil der Arbeit wird untersucht inwieweit die Gestenerkennung von den extrahierten Handinformationen der vorgestellten Verfahren zur Handanalyse auf Punktwolken profitieren kann.

Durch die methodischen Anteile dieser Arbeit können die potentiellen Vorteile der Analyse dreimensionaler geometrischer Domänen durch spezielle Deep-Learning Verfahren herausgestellt werden. In diesem Kontext konnten besonders die mit der Hand verknüpften Problematiken von den reichen räumlichen Informationen der Modalitäten profitieren. Da ein breites Spektrum von Aufgaben innerhalb dieser Arbeit abgdeckt wurde, ist die Ergänzung vieler Bereiche durch die vorgestellten Techniken denkbar, insbesondere aber liefern die Beiträge bei Anwendungen für die gestenbasierte Mensch-Maschine Interaktion einen klaren Mehrwert.
\cleardoublepage

\tableofcontents
%\newpage


%-----------------------------------------------------------
%   Chapters
%-----------------------------------------------------------
\cite{Alldieck17}
--diseases



\section{Generalization in Learning}

\section{Generalization in Deep Learning}

\section{Deep Learning in Medical Image processing}
    \subsection{Data Acquisition}
        \subsubsection{Scanners}

    \subsection{Data Curation}
    \subsection{Model Building}
        \subsubsection{Convolutional Models}
            Kernels

    \subsection{Model Application}

    \subsection{Outcome Evaluation}

% \input{sections/3_hand_pose_estimation}
% \input{sections/4_scene_flow_estimation}
% \input{sections/5_hand_tracking}
% \input{sections/6_gesture_recognition}
\input{sections/7_conclusion}

%-----------------------------------------------------------
%   TODO
%-----------------------------------------------------------
% - Zusammenfassung / Abstract	[1]
% x - Gesture Recognition Kapitel schreiben [8]
% x Introduction [4]
% x Summary / Conclusion
%    x Contributions [1]
%    x Research Findings [1]
%    x Limitations & Outlook [1]
% x - Cleanup:
% x   - Scene flow estimation [2]
% x   - Hand tracking [2]
% - Background: gesture recognition [1]
% total [20]
%
% Einreichung: 24.01.24 oder 27.03.24
%-----------------------------------------------------------
%   References
%-----------------------------------------------------------
\renewcommand\bibname{References}
\printbibliography[heading=bibintoc]

% \begin{refsection}
% \newrefcontext[sorting=ymdnt]
% \renewbibmacro*{begentry}{\makebox[2.9em][l]{\textbullet}}
% \nocite{*}
% \boldname{Bigalke}{Alexander}{A.}
% \printbibheading[title={List of Publications},heading=bibintoc]
% This list contains journal articles and conference papers published during the work on this dissertation.\\
% \printbibliography[title={Journal Articles as First Author}, keyword=journals_first_author,heading=subbibliography]
% \printbibliography[title={Conference Papers as First Author},keyword=proceedings_first_author,heading=subbibliography]
% \printbibliography[title={Journal Articles and Conference Papers as Co-Author},keyword=proceedings_co_author,heading=subbibliography]
% \end{refsection}
%
% \cleardoublepage
%
% \addcontentsline{toc}{chapter}{Curriculum Vitae}
% \markboth{}{}
% \KOMAoption{headsepline}{false}
% \includepdf[pages=-,pagecommand={}]{cv/build/cv.pdf}

%--------------------
% Ende des Dokuments
%--------------------

\end{document}